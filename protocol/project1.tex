\documentclass[a4paper,12pt]{article}
\usepackage[utf8]{inputenc}
\usepackage[english]{babel}
\usepackage{amssymb,amsmath,graphicx,listings}
\usepackage{uniinput}
\usepackage[left=2cm,right=2cm,top=2cm,bottom=2cm]{geometry}
\renewcommand{\familydefault}{\sfdefault}
\newcommand{\leadingzero}[1]{\ifnum #1<10 0\the#1\else\the#1\fi}
\newcommand{\mytoday}{\leadingzero{\day}.\leadingzero{\month}.\the\year}
\newcommand{\code}[1]{\textbf{#1}}

\lstset{language=matlab, numbers=left,numberstyle=\footnotesize, basicstyle=\footnotesize}


\begin{document}
\title{Semester project TMA4215}
\author{10068\\ and\\ 728387}
\date{\mytoday}
\maketitle
\newpage


\section{Task}
We consider minimization problems of the type
\begin{align*}
\min_{{\bf x} \in \mathbb{R}^n} g({\bf x}),\,\, g({\bf x}):=-{\bf b}^T{\bf x} + \frac12{\bf x}^TH{\bf x}+\frac1{12}{\bf x}^TC({\bf x}){\bf x},
\end{align*}
here ${\bf b} \in \mathbb{R}^n$ and, $H$ is a $n \times n$ symmetric and positive definite matrix and $C(\bf{x})$ is a diagonal
matrix with diagonal entries $c_i x^2_i, i = 1, . . . , n$. 
Here $c_i > 0$ are the components of a vector $\bf{c} \in \mathbb{R}^n$ 
and $x_i$ are the components of $\bf{x}$. 
\section{Generation of the data}
The data is generated in the function \code{data}.
$\bf{b}$ and $\bf{c}$ should be of the same dimension and $H$ should be a symmetric matrix which fits to the vectors.
Line 6 can be used to check the dimensions of the input data, but this costs a lot of resources, because \code{data} is often called. 

% what is the data

\begin{lstlisting}
function [b, H, c] = data
	b = [1; 0];
	c = [1; 1];
	H = [1, 1;
	     1, 1];
	% b' * H * c;
end
\end{lstlisting}

\section{Function, gradient and Hessian of $g$}

\begin{lstlisting}
function [ g ] =  ( X )
	[ b, H, c ] = data;
	dim = size(H,1);
	C = zeros ( dim, dim );
	for i = 1 : dim
		C(i,i) = c(i) * X(i) * X(i);
	end
	g = - b' * X + 0.5 * X' * H * X + 1/12 * X' * C * X;
end
\end{lstlisting}

\begin{align}\label{grad}
{\bf\nabla} g = -{\bf b} + H{\bf x} + \frac13C{\bf x}
\end{align}

\begin{lstlisting}
function [ nablaG ] = grad( X )
	[ b, H, c ] = data;
	dim = size ( H, 1 );
	for i = 1 : dim
	    C(i,i) = c(i) * X(i) * X(i);
	end
	nablaG = - b + H * X + 1/3 * C * X;
end
\end{lstlisting}

\begin{align}\label{hess}
{\bf\nabla}^2 g =  H + C
\end{align}

\begin{lstlisting}
function [ HessG ] = hessian( X )
	[ ~, H, c ] = data;
	dim = size ( H, 1 );
	C = zeros ( dim, dim );
	for i = 1 : dim
		C(i,i) = c(i) * X(i) * X(i);
	end
	HessG = H + C;
end
\end{lstlisting}


\section{Existence of minimum}
Let $u\in R^n$ be an arbitrary vector, except $\vec{0}$ then
$$u(\nabla^2g(x))u = u (H+C) u = u H u + uCu = uHu + \sum_{i=1}^n c_iu_i^2x_i^2.$$
Since $H$ is positive definite, $uHu>0$ and because $c_i>0$, $uCu>0$, so 
$$u(\nabla^2g(x))u>0$$ 
That means that the Hessian of $g$ is positive definite.
\section{Plot for $n=2$}

\section{Steepest decent method}
% 
\begin{lstlisting}
function [ iterations, X, Y, Z ] = steepest( Xk, alpha )
	tol = 1e-6;
	maxiterations = 1000;
	[~,H,~] = data;
	dim = size(H, 1);
	X1 = zeros(1, maxiterations);
	X2 = zeros(1, maxiterations);
	Y = zeros(1, maxiterations);
	Z = zeros(1, maxiterations);
	X1(1) = Xk(1);
	X2(1) = Xk(2);
	condition = 1;
	norm_old = norm ( grad ( Xk) );
	Y(1) = norm_old;
	Z(1) = problem(Xk);
	while condition
		maxiterations = maxiterations - 1;
		Xk = Xk - alpha * grad ( Xk );
		if dim == 2
				X1(1000-maxiterations+1) = Xk(1);
				X2(1000-maxiterations+1) = Xk(2);
		end
		residual = norm ( grad ( Xk ) ) / norm_old;
		Y(1000-maxiterations+1) = residual;
		Z(1000-maxiterations+1) = problem(Xk);
		condition = (maxiterations > 0) && ( residual > tol);
	end
	X1 = X1(1:1000-maxiterations+1);
	X2 = X2(1:1000-maxiterations+1);
	X = [X1;X2];
	iterations = (1:1:1000-maxiterations+1);
	Z = Z(1:1000-maxiterations+1);
	Y = Y(1:1000-maxiterations+1);
end
\end{lstlisting}

\section{Equivalence of steepest decent method and forward Euler method}
No idea.
\section{Improved steepest decent method}
To find the optimal $α$,
$$
g\left({\bf x}^{(k+1)}\right) = g \left({\bf x}^{(k)} - α^{(k)} \nabla g({\bf x}^{(k+1)})\right)
$$
has to be minimal, so $\frac{∂}{∂αo g({\bf x}^{(k+1)})$  has to be zero.
This leads to the equation
\begin{align*}
&\left( {\bf b}{\bf \nabla} g - {\bf x}H{\bf x} - \frac13 {\bf x}  H  {\bf \nabla} g\right)
 + \left( ({\bf \nabla} g) ( H + C) {\bf \nabla} g \right)α
\\+ &\left( - \sum_{i=1}^n c_i  x_i  ({\bf \nabla} g)_i^3 \right)α^2
+ \left( \sum_{i=1}^n c_i  ({\bf \nabla} g)_i^4\right)α^3 = 0
\end{align*}
\begin{lstlisting}

\end{lstlisting}

\section{Newton method}
\begin{lstlisting}

\end{lstlisting}

\section{Combination of steepest decent methode and Newton method}
\begin{lstlisting}

\end{lstlisting}


\end{document}
